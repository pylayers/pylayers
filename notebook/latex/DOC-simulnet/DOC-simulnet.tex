

%% This file was auto-generated by IPython.
%% Conversion from the original notebook file:
%%
\documentclass[final]{IEEEtran}

%% This is the automatic preamble used by IPython.  Note that it does *not*
%% include a documentclass declaration, that is added at runtime to the overall
%% document.
\usepackage[english]{babel}
\usepackage{amsmath}
\usepackage{amssymb}
\usepackage{graphicx}
\usepackage{algorithm}
\usepackage{algorithmic}
\usepackage{color}
\usepackage{subfig}
\usepackage{multirow}
\usepackage{ucs}
\usepackage[utf8x]{inputenc}

% needed for markdown enumerations to work
\usepackage{enumerate}
% title
\title{DOC-simulnet}
\author{Bernard Uguen}
% Slightly bigger margins than the latex defaults
\usepackage{geometry}
\geometry{verbose,tmargin=3cm,bmargin=3cm,lmargin=2.5cm,rmargin=2.5cm}

% Define a few colors for use in code, links and cell shading
\usepackage{color}
\definecolor{orange}{cmyk}{0,0.4,0.8,0.2}
\definecolor{darkorange}{rgb}{.71,0.21,0.01}
\definecolor{darkgreen}{rgb}{.12,.54,.11}
\definecolor{myteal}{rgb}{.26, .44, .56}
\definecolor{gray}{gray}{0.45}
\definecolor{lightgray}{gray}{.95}
\definecolor{mediumgray}{gray}{.8}
\definecolor{inputbackground}{rgb}{.95, .95, .85}
\definecolor{outputbackground}{rgb}{.95, .95, .95}
\definecolor{traceback}{rgb}{1, .95, .95}

% Framed environments for code cells (inputs, outputs, errors, ...).  The
% various uses of \unskip (or not) at the end were fine-tuned by hand, so don't
% randomly change them unless you're sure of the effect it will have.
\usepackage{framed}

% remove extraneous vertical space in boxes
\setlength\fboxsep{0pt}

% codecell is the whole input+output set of blocks that a Code cell can
% generate.

% TODO: unfortunately, it seems that using a framed codecell environment breaks
% the ability of the frames inside of it to be broken across pages.  This
% causes at least the problem of having lots of empty space at the bottom of
% pages as new frames are moved to the next page, and if a single frame is too
% long to fit on a page, will completely stop latex from compiling the
% document.  So unless we figure out a solution to this, we'll have to instead
% leave the codecell env. as empty.  I'm keeping the original codecell
% definition here (a thin vertical bar) for reference, in case we find a
% solution to the page break issue.

%% \newenvironment{codecell}{%
%%     \def\FrameCommand{\color{mediumgray} \vrule width 1pt \hspace{5pt}}%
%%    \MakeFramed{\vspace{-0.5em}}}
%%  {\unskip\endMakeFramed}

% For now, make this a no-op...

\newenvironment{traceback}{%
   \def\FrameCommand{\colorbox{traceback}}%
   \MakeFramed{\advance\hsize-\width \FrameRestore}}
 {\endMakeFramed}

% Use and configure listings package for nicely formatted code
\usepackage{listingsutf8}
\lstset{
  language=python,
  inputencoding=utf8x,
  extendedchars=\true,
  aboveskip=\smallskipamount,
  belowskip=\smallskipamount,
  xleftmargin=2mm,
  breaklines=true,
  basicstyle=\small \ttfamily,
  showstringspaces=false,
  keywordstyle=\color{blue}\bfseries,
  commentstyle=\color{myteal},
  stringstyle=\color{darkgreen},
  identifierstyle=\color{darkorange},
  columns=fullflexible,  % tighter character kerning, like verb
}

% The hyperref package gives us a pdf with properly built
% internal navigation ('pdf bookmarks' for the table of contents,
% internal cross-reference links, web links for URLs, etc.)
\usepackage{hyperref}
\hypersetup{
  breaklinks=true,  % so long urls are correctly broken across lines
  colorlinks=true,
  urlcolor=blue,
  linkcolor=darkorange,
  citecolor=darkgreen,
  }

% hardcode size of all verbatim environments to be a bit smaller
\makeatletter 
\g@addto@macro\@verbatim\small\topsep=0.5em\partopsep=0pt
\makeatother 

% Prevent overflowing lines due to urls and other hard-to-break entities.
\sloppy
\graphicspath{{images/}{figures/}}




\begin{document}
\pagestyle{empty}
\maketitle


\section{A 2 minutes tutorial}


This notebbok explains how to run a simation from the simulnet module.

\subsection{Running the simulation
}Once the following configuration files have been filled : simulnet.ini
agent.ini emsolver.ini communication.ini

The simulation can be started with the method \verb!runsimul!

\begin{figure}[htp]

\begin{verbatim}
Layout graphs are loaded from  /home/uguen/Bureau/P1 /struc
wait 0.1
wait 0.1
Processing save results, please wait

\end{verbatim}

\end{figure}\subsection{Analysis of the saved results
}Data are stored into \verb!S.save.save! dictionnary. If asked in
\verb!simulnet.ini!, those data are stored in a \verb!matfile! in `

\begin{figure}[htp]

\begin{verbatim}
Node #  1

emitted power {'rat1': 0}
sensitivity node {'rat1': -80}
type ag

At time stamp 0:
true position [ 18.90762689   2.53145395]
estimated position [ nan  nan]

On rat1
Received powers  [-90.7282282   3.       ]
TOA  [ 63.25263999   0.3       ]

\end{verbatim}

\end{figure}\subsection{Description of inner organization of the Simulation object
}\subsubsection{list of involved agents
}All agents (mobile and anchors) are gathered in a list of agents. Notice
that anchors are static agents.

\begin{figure}[htp]


\begin{verbatim}
[<pylayers.mobility.agent.Agent at 0x36a2790>,
 <pylayers.mobility.agent.Agent at 0x36a2950>,
 <pylayers.mobility.agent.Agent at 0x36a2890>,
 <pylayers.mobility.agent.Agent at 0x36a2810>]
\end{verbatim}


\end{figure}All moving agents have the following mechanical attributes which have an
influence on mobility.

\begin{figure}[htp]

\begin{verbatim}
Agent ID: 1
Agent name: Person_ID1
Agent mass: 80

position vector: (30.0533, 2.7772, 0.0000)
velocity vector: (0.1494, 0.8244, 0.0000)
acceleration vector: (-0.0915, 0.4152, 0.0000)

waiting time in room: 1.0
coordinates of its target: (30.1670, 4.9950, 0.0000)
a list coordinates of its intermediate target: [(18.900000000000002, 2.4975000000000005), (19.460999999999999, 4.9950000000000001), (30.167000000000002, 4.9950000000000001), (29.700000000000006, 2.4974999999999996)]

\end{verbatim}

\end{figure}\section{Network Attributes}


The network is a graph:

\begin{itemize}
\item
  Nodes of the graph represent Agents or access points
\item
  Edges of the graph represent radio link between nodes
\end{itemize}

The node is a dictionnary which contains the following keys :

\begin{itemize}
\item
  `PN' : Personnal Network ( described in the following)
\item
  `RAT' : A list of RAT of which it belongs
\item
  `p' : true position
\item
  `pe' : estimated position if it has been computed by the node itself (
  cf. location tutorial )
\item
  't' : A time stamp
\item
  `type': Its type ( `ag' : for agent or `ap' for access point )
\end{itemize}
example with node `1'

Each node are link by the edge of the graph

The edge is a dictionnary which contains the following keys :

example with node `1'




\end{document}

